% Isaac J. Yeaton
% Oct 5, 2012
%
% LaTeX template for general report writing

% top matter
\documentclass[11pt, letterpaper]{article}

% load packages
%\usepackage{showkeys}   % show labels in document
%\usepackage{paralist}   % in-paragraph lists
\usepackage{graphicx}   % for including figures
\usepackage{epstopdf}   % enable .eps images when using pdflatex
\usepackage{hyperref}   % including links within document
\usepackage{amsmath}    % more featurefull math tools
\usepackage{amssymb}    % even more math symbols
\usepackage{geometry}   % specify page dimensions
\usepackage{siunitx}    % for including units
\usepackage{listings}   % formatted code
\usepackage{appendix}   % actual appendix environment
\usepackage{varioref}   % smart page, figure, table, and equation referencing
\usepackage{wrapfig}    % wrap figures/tables in text (i.e., Di Vinci style)
\usepackage{fancyvrb}   % extended verbatim environments
\usepackage{subfigmat}  % matrices of similar subfigures, aka small mulitples
\usepackage{nomencl}    % inserting nomenclature
\usepackage{pdfpages}   % including external pdfs
\usepackage{subfigure}  % adding subfigures
\usepackage{longtable}  % multipage tables
\usepackage{setspace}   % change spacing for document
\usepackage[all]{hypcap}        % make links point to top of image
\usepackage{threeparttable}     % tables with footnotes
%\usepackage[parfill]{parskip}   % modify paragraph breaks and spacing
\usepackage[version=3]{mhchem}  % chemical equations and notation

% modify the spacing
\frenchspacing   % modify the spacing between words
%\onehalfspacing  % put more space between lines

% modify some packages
%\geometry{margin=1in}
\geometry{letterpaper}
\hypersetup{colorlinks=true, linkcolor=black,
            citecolor=black, urlcolor=black}

% include code
\lstset{language=Python, basicstyle=\footnotesize, frame=single,
        numbers=left}

% bibliography
\usepackage[square, numbers, sort&compress, comma]{natbib}
\bibliographystyle{plainnat}

%% -------------------------------------------------------------------------

% user defined commands
\newcommand{\bibproj}[0]{../bibs/tmp1}  %TODO add a bib file here...
\newcommand{\bibbook}[0]{../bibs/tmp2}  %TODO add another bib file here...
\newcommand{\fig}[1]{figure~\ref{#1}}
\newcommand{\sect}[1]{section~\ref{#1}}
\newcommand{\eqn}[1]{equation~\eqref{#1}}
\newcommand{\tab}[1]{table~\ref{#1}}
\newcommand{\comment}[1]{}  % easy way to block out text

% shortcuts


%% -------------------------------------------------------------------------

% headings to be displayed
\pagestyle{myheadings}
\markright{Isaac~Yeaton -- \LaTeX\ template}

% document parameters
\title{\LaTeX\ template}
\author{Isaac J.~Yeaton}
\date{October 5, 2012}

%% -------------------------------------------------------------------------

\begin{document}
\maketitle
\thispagestyle{empty}

%% -------------------------------------------------------------------------

\begin{abstract}
	Include the abstract here...
\end{abstract}

% TOC and LOF
\tableofcontents
\listoffigures

%% -------------------------------------------------------------------------

\section{Introduction}

When writing Dr, use Dr.\ Lowe (with a \verb|.\|) to prevent excess space after the 
period.  Also, put a \verb|~| in you don't want a line to break between the two works.

Start new lines when you reach the end of the editor window...don't just keep
writing or you will have a paragraph that spans multiple lines, making it
harder to edit when using a different editor.

Look at the other folders for example of including figures and bibliography.

\section{Some sample environments}

Using \texttt{itemize}:
%
\begin{itemize}
	\item This
	\item That
	\begin{itemize}
		\item subitem
		\item another
	\end{itemize}
	\item end of list
\end{itemize}

Using \texttt{enumerate}:
%
\begin{enumerate}
	\item This
	\item That
	\begin{itemize}
		\item subitem
		\item another
	\end{itemize}
	\item end of list
\end{enumerate}

%\begin{figure}
%	\centering
%	\includegraphics[scale=1]{./figs/tmp.png}
%	\caption{Excellent caption.}
%	\label{fig:fig_desc}
%\end{figure}

\section{Conclusion}


\end{document}
% Isaac J. Yeaton
% Oct 5, 2012
%
% LaTeX template for general report writing

% top matter
\documentclass[11pt, letterpaper]{article}

% load packages
%\usepackage{showkeys}   % show labels in document
%\usepackage{paralist}   % in-paragraph lists
\usepackage{graphicx}   % for including figures
\usepackage{epstopdf}   % enable .eps images when using pdflatex
\usepackage{hyperref}   % including links within document
\usepackage{amsmath}    % more featurefull math tools
\usepackage{amssymb}    % even more math symbols
\usepackage{geometry}   % specify page dimensions
\usepackage{siunitx}    % for including units
\usepackage{listings}   % formatted code
\usepackage{appendix}   % actual appendix environment
\usepackage{varioref}   % smart page, figure, table, and equation referencing
\usepackage{wrapfig}    % wrap figures/tables in text (i.e., Di Vinci style)
\usepackage{fancyvrb}   % extended verbatim environments
\usepackage{subfigmat}  % matrices of similar subfigures, aka small mulitples
\usepackage{nomencl}    % inserting nomenclature
\usepackage{pdfpages}   % including external pdfs
\usepackage{subfigure}  % adding subfigures
\usepackage{longtable}  % multipage tables
\usepackage{setspace}   % change spacing for document
\usepackage[all]{hypcap}        % make links point to top of image
\usepackage{threeparttable}     % tables with footnotes
%\usepackage[parfill]{parskip}   % modify paragraph breaks and spacing
\usepackage[version=3]{mhchem}  % chemical equations and notation

% modify the spacing
\frenchspacing   % modify the spacing between words
%\onehalfspacing  % put more space between lines

% modify some packages
\geometry{margin=1in}
\hypersetup{colorlinks=true, linkcolor=black,
            citecolor=black, urlcolor=black}

% include code
\lstset{language=Python, basicstyle=\footnotesize, frame=single,
        numbers=left}

% bibliography
\usepackage[square, numbers, sort&compress, comma]{natbib}
\bibliographystyle{plainnat}

%% -------------------------------------------------------------------------

% user defined commands
\newcommand{\bibproj}[0]{../bibs/tmp1}  %TODO add a bib file here...
\newcommand{\bibbook}[0]{../bibs/tmp2}  %TODO add another bib file here...
\newcommand{\fig}[1]{figure~\ref{#1}}
\newcommand{\sect}[1]{section~\ref{#1}}
\newcommand{\eqn}[1]{equation~\eqref{#1}}
\newcommand{\tab}[1]{table~\ref{#1}}
\newcommand{\comment}[1]{}  % easy way to block out text

% shortcuts


%% -------------------------------------------------------------------------

% headings to be displayed
\pagestyle{myheadings}
\markright{I.~Yeaton -- \LaTeX\ template}

% document parameters
\title{\LaTeX\ template}
\author{Isaac J.~Yeaton}
\date{October 5, 2012}

%% -------------------------------------------------------------------------

\begin{document}
\maketitle
\thispagestyle{empty}

%% -------------------------------------------------------------------------

\begin{abstract}
	Include the abstract here...
\end{abstract}

% TOC and LOF
\tableofcontents
\listoffigures

%% -------------------------------------------------------------------------

\section{Introduction}

When writing Dr, use Dr.\ Lowe (with a \verb|.\|) to prevent excess space after the 
period.  Also, put a \verb|~| in you don't want a line to break between the two works.

Start new lines when you reach the end of the editor window...don't just keep
writing or you will have a paragraph that spans multiple lines, making it
harder to edit when using a different editor.

Look at the other folders for example of including figures and bibliography.

\section{Some sample environments}

Using \texttt{itemize}:
%
\begin{itemize}
	\item This
	\item That
	\begin{itemize}
		\item subitem
		\item another
	\end{itemize}
	\item end of list
\end{itemize}

Using \texttt{enumerate}:
%
\begin{enumerate}
	\item This
	\item That
	\begin{itemize}
		\item subitem
		\item another
	\end{itemize}
	\item end of list
\end{enumerate}

%\begin{figure}
%	\centering
%	\includegraphics[scale=1]{./figs/tmp.png}
%	\caption{Excellent caption.}
%	\label{fig:fig_desc}
%\end{figure}

\section{Conclusion}


\end{document}

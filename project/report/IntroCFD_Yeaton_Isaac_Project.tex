% Isaac J. Yeaton
% Dec 12, 2012
%
% Advanced Intro to CFD, Final project report

% top matter
\documentclass[10pt, letterpaper]{article}

% load packages
%\usepackage{showkeys}   % show labels in document
%\usepackage{paralist}   % in-paragraph lists
\usepackage{graphicx}   % for including figures
\usepackage{epstopdf}   % enable .eps images when using pdflatex
\usepackage{hyperref}   % including links within document
\usepackage{amsmath}    % more featurefull math tools
\usepackage{amssymb}    % even more math symbols
\usepackage{geometry}   % specify page dimensions
\usepackage{siunitx}    % for including units
\usepackage{listings}   % formatted code
\usepackage{appendix}   % actual appendix environment
\usepackage{varioref}   % smart page, figure, table, and equation referencing
\usepackage{wrapfig}    % wrap figures/tables in text (i.e., Di Vinci style)
\usepackage{fancyvrb}   % extended verbatim environments
\usepackage{nomencl}    % inserting nomenclature
\usepackage{pdfpages}   % including external pdfs
%\usepackage{subfigure}  % adding subfigures
\usepackage{longtable}  % multipage tables
\usepackage{setspace}   % change spacing for document
\usepackage[all]{hypcap}        % make links point to top of image
\usepackage{threeparttable}     % tables with footnotes
%\usepackage[parfill]{parskip}   % modify paragraph breaks and spacing
\usepackage[version=3]{mhchem}  % chemical equations and notation

\usepackage{caption}
\usepackage{subcaption}

% modify the spacing
\frenchspacing   % modify the spacing between words
%\onehalfspacing  % put more space between lines

% modify some packages
%\geometry{margin=1in}
\geometry{letterpaper}
\hypersetup{colorlinks=true, linkcolor=black,
            citecolor=black, urlcolor=black}

% include code
\lstset{language=Python, basicstyle=\footnotesize, frame=single,
        numbers=left}

% bibliography
\usepackage[square, numbers, sort&compress, comma]{natbib}
\bibliographystyle{plainnat}

%% -------------------------------------------------------------------------

% user defined commands
\newcommand{\p}{\partial}
\newcommand{\fig}[1]{figure~\ref{#1}}
\newcommand{\sect}[1]{section~\ref{#1}}
\newcommand{\eqn}[1]{equation~\eqref{#1}}
\newcommand{\tab}[1]{table~\ref{#1}}
\newcommand{\comment}[1]{}  % easy way to block out text
%http://tex.stackexchange.com/questions/44545/newcommand-gives-errors-in-math-mode-with-or-without-arguments
\newcommand{\pf}[3][]{% \deriv[<order>]{<func>}{<var>}
  \ensuremath{\frac{\partial^{#1} {#2}}{\partial {#3}^{#1}}}}
\newcommand{\mesh}[1]{${#1} \times {#1}$}
%\newcommand{\pf}[2]{$\frac{\partial {#1}}{\partial {#2}}$}
%\newcommnad{\pfn}[3]{$\frac{\partial^{#3} {#1}}{\partial {#2}^{#3}}$}
%\newcommnad{\lmaxx}[1]{$|\lambda_x|_{\mathrm{max}}$}
%\newcommnad{\lmaxy}[1]{$|\lambda_y|_{\mathrm{max}}$}
%\newcommnad{\C4}{$C^{(4)}$}
%\newcommand{\b2}{$\beta^2$}

% shortcuts


%% -------------------------------------------------------------------------

% headings to be displayed
\pagestyle{myheadings}
\markright{I.~Yeaton -- Numerical Study of Incompressible Lid-Driven Cavity Flow}

% document parameters
\title{Numerical Study of Incompressible Lid-Driven Cavity Flow}
\author{Isaac J.~Yeaton}
\date{December 12, 2012}

%% -------------------------------------------------------------------------

\begin{document}
\maketitle
\thispagestyle{empty}

%% -------------------------------------------------------------------------

\begin{abstract}
	Two-dimensional, lid-driven cavity flow was solved using CFD code for 
	Reynolds numbers
	of 100, 500, and 1000 using the 2D~incompressible Navier-Stokes equations
	with time derivative preconditioning (temperature and viscosity were assumed
	constant).  The code was checked with the method of manufactured solutions
	for both a point Jacobi and symmetric Gauss-Seidel numerical schemes.  This
	was done at a Reynolds number of 10 and used to show the observed order of
	accuracy approaches second-order with mesh refinement.  Symmetric meshes
	of \mesh{65}, \mesh{129}, \mesh{257}, and \mesh{513} nodes were run. 
	Additionally, textsc{OpenFOAM} was used to check the lid driven cavity case
	and the comparison shows very similar results, with minor discrepancies
	for the pressure.  A study of mesh size, $\kappa$, and 
	$\mathrm{C^{(4)}}$ constants
	for time derivative preconditioning and artificial viscosity was performed for 
	the manufactured case and their effect on the solution	discussed.
\end{abstract}

%% -------------------------------------------------------------------------

\section{Introduction}

The 2D incompressible Navier-Stokes equations with time derivative preconditioning
and artificial viscosity are given by
%
\begin{align} \label{eqn:2dns}
	\frac{1}{\beta^2} \pf{p}{t} + \rho \pf{u}{x} + \rho \pf{v}{y} &= S \\
	\rho \pf{u}{t} + \rho u \pf{u}{x} + \rho v \pf{u}{y} + \pf{p}{x} &= 
		\mu \pf[2]{u}{x} + \mu \pf[2]{u}{y} \\
	\rho \pf{v}{t} + \rho u \pf{v}{x} + \rho v \pf{v}{y} + \pf{p}{y} &= 
		\mu \pf[2]{v}{x} + \mu \pf[2]{v}{y},
\end{align}
%
where $\beta^2 = \mathrm{max}(u^2 + v^2, \kappa \mathrm{U_{lid}}^2)$ is the time 
derivative preconditioning term.  Acceptable values of $\kappa$ are between 0.001 
and 0.9.  In the above equation, $S$ is the artificial viscosity term given by
%
\begin{equation} \label{eqn:S}
	S = -\frac{|\lambda_x|_{\mathrm{max}} \mathrm{C}^{(4)}}{\beta^2} \pf[4]{p}{x}
	    -\frac{|\lambda_y|_{\mathrm{max}} \mathrm{C}^{(4)}}{\beta^2} \pf[4]{p}{y},
\end{equation}
%
where the largest eigenvalues in $(x,t)$ and $(y,t)$ are 
$|\lambda_x|_{\mathrm{max}}$ and
$|\lambda_y|_{\mathrm{max}}$, respectively.  The eigenvalues for this problem are 
$\lambda_x = 0.5(|u| + \sqrt{u^2 + 4\beta^2})$ and 
$\lambda_y = 0.5(|v| + \sqrt{v^2 + 4\beta^2})$.  The $\mathrm{C}^{(4)}$ constant
generally lies in the range of $1/128 \leq \mathrm{C}^{(4)} \leq 1/16$.  Note
the for the manufactured solution code verification, a source term was added to the
right-hand side of each equation ($f_\mathrm{mass}(x,y)$, $f_\mathrm{xmtm}(x,y)$,
and $f_\mathrm{ymtm}(x,y)$, respectively).

%% -------------

\subsection{Problems specification and solution technique}

The cavity studied is square, with side lengths of \SI{0.05}{m} (also the characteristic length $L$), with the top surface moving at 
$\mathrm{U_{lid}} = \SI{1}{m/s}$.  The
density is \SI{1.0}{kg/m^3}.  For this problem, the Reynolds number is
defined as
%
\begin{equation} \label{Re}
	Re = \frac{\rho \mathrm{U_{lid}} L}{\mu}.
\end{equation}
%
Three different Reynolds numbers of 100,~500, and 1000 run have the effect of
changing the dynamic viscosity $\nu$ of the fluid.  Since the
pressure difference is only used when solving the discretized equations (shown below),
the pressure was rescaled  to a reference value of $\mathrm{p_{ref}} = 0.80133$
at the center of the cavity after every iteration.  This prevents drift in the
solution by anchoring it to a specific value.

Since we are interested in only the steady-state solution, the discretized equations
were marched in pseudo-time until the relative iterative residuals (calculated
with the $L_2$ norm) were below \num{1e-10} for all quantities ($p$, $u$, $v$).
The iterative residuals is
%
\begin{equation} \label{eqn:resid}
	R_{i,j}^n = \sqrt{\frac{1}{N} \sum_{i,j}^N 
		\frac{u_{i,j}^n - u_{i,j}^{n-1}}{dt_{i,j}}}
\end{equation}
%
and is normalized by the iterative residual at iteration 4 to get the relative
iterative residual.  This is what was used to monitor convergence.

Local time stepping was used at each node to get maximum possible step at
each iteration.  The time step is chosen based on the minimum of the
diffusive ($t_d$) and convection ($t_c$) stability criteria.  The diffusive
time scale is
%
\begin{equation*}
	\Delta t_d \leq \frac{\Delta x \Delta y}{4 \nu},
\end{equation*}
%
and the convective time scale is
\begin{equation*}
	\Delta t_c \leq \frac{\min(\Delta x, \Delta y)}{|\lambda|_\mathrm{max}},
\end{equation*}
%
where $\Delta x$ and $\Delta y$ are the cell lengths.  The time step at each node
was then chosen using $\Delta	 t = \mathrm{CFL} \cdot \min{\Delta t_d, \Delta t_c}$,
where $\mathrm{CFL}$ is chosen so that the method is stable (and can be greater
than unity depending on the solution method and Reynolds number).

The boundary conditions for this problem are no slip along the walls.  Therefore
on the bottom and sides, $u=v=0$, and along the top $u=U_\mathrm{lid}$, $v=0$.
The pressure at the wall was found using linear interpolation from the
interior.

The discretization error and observed order of accuracy were determined from
the manufactured solution results.  The discretization error is found by
taking the norm of difference between the exact and numerical solutions of
the equations.  The observed order is then found using
%
\begin{equation} \label{eqn:ooa}
	\hat{p} = \frac{\ln(DE_2/DE_1)}{\ln r},
\end{equation}
%
where subscript 2 and 1 are the coarse and fine mesh, respectively and $r$
is the grid refinement factor (this is two for all work presented here since
grids were systematically refined).  Monitoring discretization error norms
and observed order of accuracy ensure the numerical schemes are working as
predicted.

Two schemes are implemented to solve the discretized equations.  The first is
point-Jacobi~(PJ), which uses central differences in space and uses information
at the previous time step to update the vector of unknowns.  The second is
symmetric Gauss-Seidel (SGS), which uses a forward and backward sweep through
the domain, using the newest information to update the vector of unknowns at
intermediate time steps.  SGS requires more computations per iterations, but
it converges faster and can be more stable (allows for higher $\mathrm{CFL}$
numbers).

%% -------------

\subsection{Code setup and data analysis pipeline}

The code used for this study was modified from the C template provided.
The relevant functions for time steps, artificial viscosity, point-Jacboi,
symmetric Gauss-Seidel, and discretization error norms was added.  The
code was compiled using the version 3.1 of the clang compiler
(\url{http://clang.llvm.org/} with \texttt{-O3} compiler optimizations for
increased speed.  This compiler was selected because of its useful debug messages
and faster speed than the classic GNU compiler collection.  Additionally, the 
code was modified so that the user only
has to change a few parameters at the top of the file and perform a 
\texttt{make run3} to run the case.  This allowed multiple runs to be performed
simultaneously.  All cases were run on an Ubuntu~12.10 machine with eight
hyperthreaded cores (i7-3630QM CPU@2.40GHz).  Each run used 100\% of a
hyper-threaded core.  The code is stored on
GitHub at \url{github.com/isaacyeaton/adv-intro-cfd-2012}.  Plotting was done 
in both \textsc{ParaView} and Python.  With this code, the symmetric Gauss-Siedel 
method used  most often because it allowed for CFL numbers greater than unity 
in most cases
and converged in fewer iterations that point Jacobi.  Note that both schemes
ran fairly quickly with this configuration (maximum of 26~minutes to run
the manufactured solution on a \mesh{257} grid).  Additionally,
\textsc{OpenFOAM-2.1.1-221db2718bbb} was used the check the solution from
the C code.

%% -------------------------------------------------------------------------

\newpage

\section{Manufactured solution results}

\subsection{Velocity and pressure profiles}

Contour plots of manufactured solution and iterative residual for different
mesh sizes and solution methods are shown in figure~\ref{fig:MMS}.  The contour
plots match those of the exact solution well (not shown because of space
considerations).  The iterative residuals are shown for u-velocity, but pressure
and v-velocity show the same trend.  Note that the SGS scheme converges in
fewer iterations than PJ for all meshes.  This is more pronounced as the 
mesh size increases.

\begin{figure}[hb]
	\centering
	\begin{subfigure}[b]{0.475\textwidth}
		\centering
		\includegraphics[width=\textwidth]{./figs/MMS_u.eps}
		\caption{u-velocity}
		\label{fig:MMS_u}
	\end{subfigure}
	~
	\begin{subfigure}[b]{0.475\textwidth}
		\centering
		\includegraphics[width=\textwidth]{./figs/MMS_v.eps}
		\caption{v-velocity}
		\label{fig:MMS_v}
	\end{subfigure}
	
	\begin{subfigure}[b]{0.475\textwidth}
		\centering
		\includegraphics[width=\textwidth]{./figs/MMS_p.eps}
		\caption{pressure}
		\label{fig:MMS_p}
	\end{subfigure}
	~
	\begin{subfigure}[b]{0.475\textwidth}
		\centering
		\includegraphics[width=.95\textwidth]{./figs/MMS_history.eps}
		\caption{iterative residuals}
		\label{fig:MMS_history}
	\end{subfigure}
	\caption{Manufactured solution results and iterative converge.  Contour
		plot were made with a mesh of \mesh{257} nodes and SGS method with
		a CLF=0.01, $\mathrm{C^{(4)}}$, and $\kappa$ value of 0.1.}
	\label{fig:MMS}
\end{figure}

\clearpage

%% -------------

\subsection{Effect of $\kappa$ constant}

The $\kappa$ constant in the time derivative preconditioning term effects how 
quickly the solution converges, by affecting the time derivative term in the
continuity equation.  This effect is shown in figure~\ref{fig:MMS_kappa}.  We see
that by increasing the $\kappa$ value, the solution converges marginally faster.
However, this effect if not very noticeable for smallest two values of
0.01 and 0.1.  There might be a more dramatic effect is point-Jacobi was used
since it generally requires more iterations to complete than SGS.  By modifying
this constant, one can modify the convergence and possible make something that
would be unstable converge.

\begin{figure}[h]
	\centering
	\includegraphics[scale=.5]{./figs/MMS_kappa.eps}
	\caption{Effect of time derivative preconditioning constant $\kappa$ on
		iterative convergence history of pressure.  Note that this parameter has
		little effect on the convergence of u- and v-velocities, so they 
		are now shown.}
	\label{fig:MMS_kappa}
\end{figure}

%% -------------

\subsection{Effect of $\mathrm{C^{(4)}}$ constant}

Since the continuity equation has no diffusion or viscous terms in it, it
is susceptible to odd-even decouple, where adjacent nodes have different numerical
behavior when using central differences.  This is alleviated by introducing
an artificial viscosity term, as in \eqn{eqn:S}.  One way to control the
strength of this is with the $\mathrm{C^{(4)}}$ constant, with higher values leading
to more coupling. The artificial viscosity term increases when there are large
pressure gradients (such as in the corners of the flow), and gets smaller when
the mesh is refined.  Since the effect of $S$ decreases as $\Delta x^3$ it has
little effect on the order of accuracy of the scheme.
This was investigated at the upper left of the cavity,
where pressure gradients are strongest.

The effect of $\mathrm{C^{(4)}}$ is shown in \fig{fig:MMS_C4}, which shows contour
plots of discretization error of pressure in the upper left corner of the flow, with
the zero contour drawn.  As $\mathrm{C^{(4)}}$ increases, we see discretization 
errors that are lower farther from the wall.  TODO: finish here

\begin{figure}
	\centering
	\begin{subfigure}[b]{0.475\textwidth}
		\centering
		\includegraphics[width=\textwidth]{./figs/MMS_C4_008.eps}
		\caption{$\mathrm{C^{(4)}} = 0.008$}
	\end{subfigure}
	~
	\begin{subfigure}[b]{0.475\textwidth}
		\centering
		\includegraphics[width=\textwidth]{./figs/MMS_C4_01.eps}
		\caption{$\mathrm{C^{(4)}} = 0.01$}
	\end{subfigure}
	
	\begin{subfigure}[b]{0.475\textwidth}
		\centering
		\includegraphics[width=\textwidth]{./figs/MMS_C4_04.eps}
		\caption{$\mathrm{C^{(4)}} = 0.04$}
	\end{subfigure}
	~
	\begin{subfigure}[b]{0.475\textwidth}
		\centering
		\includegraphics[width=\textwidth]{./figs/MMS_C4_0625.eps}
		\caption{$\mathrm{C^{(4)}} = 0.0625$}
	\end{subfigure}
	\caption{Effect of $\mathrm{C^{(4)}}$ constant on normalized discretization
		error for pressure for the manufactured solution using a mesh of
		\mesh{129} nodes and $\kappa = 0.1$.  The zero contour is also drawn.}
	\label{fig:MMS_C4}
\end{figure}

%% -------------

\subsection{Discretization errors}

The discretization error norms are calculated as in \eqn{eqn:ooa} using
the $L_1$, $L_2$, and $L_\infty$ norms and plotted against grid refinement 
factor $h$ in \fig{fig:MMS_accuracy}.  The discretization norms follow the expected
trend of decreasing linearly on a log-log plot with decreasing $h$ 
(larger number of nodes).
The observed order of accuracy also shows the expected trend.  As the
grid is refined more, the norms decrease to two, indicating a second-order
accurate scheme.  Note that the $L_\infty$ for pressure has a lower observed order.
This is due to the larger pressure singularity at the corners of the flow
with decreasing mesh size.

\begin{figure}
	\centering
	\begin{subfigure}[b]{0.475\textwidth}
		\centering
		\includegraphics[width=\textwidth]{./figs/MMS_DE_norms.eps}
		\caption{DE norms}
	\end{subfigure}
	~
	\begin{subfigure}[b]{0.475\textwidth}
		\centering
		\includegraphics[width=\textwidth]{./figs/MMS_OOA.eps}
		\caption{Order of accuracy}
	\end{subfigure}
	\caption{Discretization errors and order of accuracy for the manufactured
		solution.}
	\label{fig:MMS_accuracy}
\end{figure}

%% -------------------------------------------------------------------------

\newpage

\section{Cavity flow}

Figure~\ref{fig:cavity_100} shows the velocity and pressure profiles for the
lid driven cavity at a Reynods number of 100 and a mesh of \mesh{257} nodes.
We see a negative u-velocity in the center of the cavity as circulating.  There
are also up and down swells in the v-velocity at the upper left and right corners,
respectively.  Note that there is not much pressure change throughout the
cavity.  The pressure at the upper corners increases with decreasing mesh size,
indicating that there is a singularity there.

\begin{figure}[hb]
	\centering
	\begin{subfigure}[b]{0.475\textwidth}
		\centering
		\includegraphics[width=\textwidth]{./figs/cavityU_Re=100.eps}
		\caption{u-velocity}
	\end{subfigure}
	~
	\begin{subfigure}[b]{0.475\textwidth}
		\centering
		\includegraphics[width=\textwidth]{./figs/cavityV_Re=100.eps}
		\caption{v-velocity}
	\end{subfigure}
	
	\begin{subfigure}[b]{0.475\textwidth}
		\centering
		\includegraphics[width=\textwidth]{./figs/cavityP_Re=100.eps}
		\caption{pressure}
	\end{subfigure}
	\caption{Driven cavity solution using a \mesh{257} nodes and SGS method with
		a CFL number of 1.2, $\mathrm{C^{(4)}}$, and $\kappa$ value of 0.1.}
	\label{fig:cavity_100}
\end{figure}

\clearpage

%% -------------

\subsection{Reynolds number effects}

Figure~\ref{fig:cavity_Re_effects} shows the Reynolds number dependence when it is 
increased from 100 to 1000.  We see that the center of recirculation
moves lower and towards the center and that its magnitude increases.  Also,
the v-velocity profiles becomes more symmetric about the center line.  Note that
as the Reynolds number increases, the recirculation zones in the lower corners
increases in size, protruding more into the center of the flow.  Also, the 
recirculation zone at the top left corner of the lid grows.  
The pressure is not plotting
because it does not change greatly with the different Reynolds numbers.

\begin{figure}
	\centering
	\begin{subfigure}[b]{0.475\textwidth}
		\centering
		\includegraphics[width=\textwidth]{./figs/cavityU_Re=100.eps}
		\caption{Re = 100}
	\end{subfigure}
	~
	\begin{subfigure}[b]{0.475\textwidth}
		\centering
		\includegraphics[width=\textwidth]{./figs/cavityV_Re=100.eps}
		\caption{Re = 100}
	\end{subfigure}
	
	\begin{subfigure}[b]{0.475\textwidth}
		\centering
		\includegraphics[width=\textwidth]{./figs/cavityU_Re=500.eps}
		\caption{Re = 500}
	\end{subfigure}
	~
	\begin{subfigure}[b]{0.475\textwidth}
		\centering
		\includegraphics[width=\textwidth]{./figs/cavityV_Re=500.eps}
		\caption{Re = 500}
	\end{subfigure}
	
	\begin{subfigure}[b]{0.475\textwidth}
		\centering
		\includegraphics[width=\textwidth]{./figs/cavityU_Re=1000.eps}
		\caption{Re = 1000}
	\end{subfigure}
	~
	\begin{subfigure}[b]{0.475\textwidth}
		\centering
		\includegraphics[width=\textwidth]{./figs/cavityV_Re=1000.eps}
		\caption{Re = 1000}
	\end{subfigure}
	\caption{Reynolds number effects for the cavity.  The left column
		    is u-velocity, and the right is v-velocity.}
	\label{fig:cavity_Re_effects}
\end{figure}

%% -------------

\subsection{Comparison to \textsc{OpenFOAM}}

\textsc{OpenFOAM} was used to check cavity results at the $\mathrm{Re}=100$
cases.  \textsc{OpenFOAM} was run for mesh sizes of \mesh{41}, \mesh{81}, and
\mesh{161}.
The \mesh{81} mesh was used to compare to the \mesh{257} case shown above in
figure~\ref{fig:cavity_100}.  This results is shown in \fig{fig:openfoam}.
The greatest discrepancy is in pressure, which is reasonable since the
\textsc{OpenFOAM} calculation was doing no pressure rescaling (in
fact it was set to zero gage pressure thoughout the domain).  However, bot the
u- and v-velocity contour plots show the same same structures and are very
close in size and magnitude.

\begin{figure}[p]
	\centering
	\begin{subfigure}[b]{0.475\textwidth}
		\centering
		\includegraphics[width=\textwidth]{./figs/compare_257x257_U_Re=100.png}
		\caption{C code, u-velocity}
	\end{subfigure}
	~
	\begin{subfigure}[b]{0.475\textwidth}
		\centering
		\includegraphics[width=\textwidth]{./figs/openfoam_u_81x81.png}
		\caption{\textsc{OpenFOAM} u-velocity}
	\end{subfigure}
	
	\begin{subfigure}[b]{0.475\textwidth}
		\centering
		\includegraphics[width=\textwidth]{./figs/compare_257x257_V_Re=100.png}
		\caption{C code, v-velocity}
	\end{subfigure}
	~
	\begin{subfigure}[b]{0.475\textwidth}
		\centering
		\includegraphics[width=\textwidth]{./figs/openfoam_v_81x81.png}
		\caption{\textsc{OpenFOAM} v-velocity}
	\end{subfigure}
	
	\begin{subfigure}[b]{0.475\textwidth}
		\centering
		\includegraphics[width=\textwidth]{./figs/compare_257x257_P_Re=100.png}
		\caption{C code, pressure}
	\end{subfigure}
	~
	\begin{subfigure}[b]{0.475\textwidth}
		\centering
		\includegraphics[width=\textwidth]{./figs/openfoam_p_81x81.png}
		\caption{\textsc{OpenFOAM} pressure}
	\end{subfigure}
	\caption{Check of the code against \textsc{OpenFOAM} for the $Re=100$ case.}
	\label{fig:openfoam}
\end{figure}

\clearpage

%% -------------------------------------------------------------------------

\section{Proof of version control}

\begin{figure}[h]
	\centering
	\includegraphics[width=.7\textwidth]{./figs/git_proof.png}
	\caption{Git diff of the C file for iteraive residual code.}
\end{figure}

\begin{figure}[h!]
	\centering
	\includegraphics[width=.7\textwidth]{./figs/git_proof1.png}
	\caption{Git diff of the C file for point-Jacobi scheme.}
\end{figure}

\clearpage

\end{document}